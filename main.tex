\documentclass{article}
\usepackage{graphicx} % Required for inserting images
\usepackage[utf8]{inputenc}
\usepackage[T1]{fontenc} 
\usepackage[polish]{babel}

\title{\textbf{Specyfikacja funkcjonalna programu dzielącego graf}}
\author{Adam Domański, Oliwier Osiński}
\date{09.03.2025}

\begin{document}

\maketitle

\section*{Cel projektu}

Celem projektu jest stworzenie programu dzielącego graf w języku C. Program ma podzielić graf określoną liczbę razy, przy jak najmniejszej liczbie usuniętych krawędzi. Różnica w liczbie wierzchołków w nowo utworzonych grafach nie może się różnić o więcej niż podany przez użytkownika margines procentowy. Wszystkie parametry mają być przyjmowane z linii poleceń. Program ma wypisywać otrzymany graf w trybie tekstowym lub binarnym w zależności od preferencji użytkownika.

\section*{Argumenty wywołania programu}
Do prawidłowego uruchomienia programu należy podać następujące argumenty:
\begin{itemize}
    \item ścieżka pliku wejściowego: ścieżka do pliku, który zawiera tekstową interpretację grafu;

    \item liczba podzieleń grafu (N): dodatnia liczba całkowita, której domyślna wartość wynosi 1;

    \item margines różnicy procentowej między powstałymi grafami (M): nieujemna liczba całkowita, której domyślna wartość wynosi 10 (wartości interpretowane w procentach);

    \item ścieżka pliku wyjściowego: ścieżkę do pliku, w którym zostanie zapisany graf po dokonaniu podziałów
    
\end{itemize}

\section*{Plik wejściowy}
Do prawidłowego dokonania podziału grafu potrzebne będą informacje o grafie, który ma zostać podzielony. Dane te mają być zapisane w formacie tekstowym w pliku txt.\\

\textbf{Plik z grafem wejściowym składa się z następujących linii:}

\begin{itemize}
  \item Maksymalna liczba węzłów w wierszu.
  \item Indeksy węzłów w poszczególnych wierszach.
  \item Wskaźniki na pierwsze indeksy węzłów w liście wierszy.
  \item Grupy węzłów połączone przy pomocy krawędzi.
  \item Wskaźniki na pierwsze węzły w poszczególnych grupach.
\end{itemize}
\textbf{Przykład grafu w pliku wejściowym:}\\
\texttt{5\\0;2;4;1;3;0;1;2;3\\0;3;5;9\\0;1;2;5;1;3;4;2;5;7;3;6;4;5;7;5;8;6;7;8 \\0;4;7;10;12;15;17;20}

\section*{Dane wyjściowe}
Wyniki operacji programu mogą zostać wyświetlone lub zapisane na dwa różne sposoby.
\begin{enumerate}
\item W domyślnym trybie tekstowym najpierw w pierwszej linii zwracana jest wartość pomyślnie podzielonych grafów, a następnie podzielone grafy w identycznym formacie jak w pliku wejściowym.

\item W trybie binarnym, gdzie ma być zwracane tylko grafy. Tryb ten nie ma z góry określonego formatu.
\end{enumerate}

Warto zaznaczyć, że w zależności od podanych flag, program może jednocześnie wyświetlić wynik w terminalu, jak i zapisać go do pliku.\\

\textbf{Program przyjmuje następujące flagi:}
\begin{itemize}
\item -o plik.out - flaga przyjmuje argument w postaci ścieżki do pliku, w którym ma zostać zapisany wynik operacji;

\item -t - flaga powoduje wyświetlenie wyniku w terminalu;

\item -b - flaga zmienia sposób wyświetlania wyniku z tekstowego na binarny
\end{itemize}

\section*{Teoria}

Program po wczytaniu grafu, dzieli go określoną przez użytkownika liczbę razy, przecinając przy tym jak najmniejszą liczbę krawędzi. Dla każdego podziału różnica procentowa wierzchołków dwóch nowoutworzonych grafów nie może być większa od marginesu procentowego podanego przez użytkownika. Jest ona wyliczana ze wzoru: \[ \frac{|A - B|}{\frac{(A + B)}{2}} \cdot 100\%\],gdzie \textbf{A} - liczba wierzchołków pierwszego z nowopostałych grafów, natomiast \textbf{B} liczba wierzchołków drugiego. Gdy graf został podzielony określoną liczbę razy lub dalszy podział jest niemożliwy, program zapisuje wyniki i kończy działanie.

\section*{Możliwe błędy podczas uruchamiania}
\textbf{Program przerywa działanie, gdy napotka nieprawidłowości we wprowadzonych danych i wyświetla odpowiedni komunikat:}

\begin{enumerate}
  \item Nieudana próba otwarcia pliku wejściowego: \textbf{Nie udało się otworzyć pliku wejściowego o podanej ścieżkce. Przerywam działanie.}

  \item Błąd w odczycie pliku wejściowego: \textbf{Dane w pliku przedstawiające graf są niepoprawne. Przerywam działanie.}
  
  \item Niepoprawna liczba podzieleń grafu: \textbf{Liczba podzieleń grafu musi być większa bądź równa 1. Przerywam działanie.}
  
  \item Niepoprawny margines różnicy procentowej: \textbf{Liczba marginesu różnicy procentowej między wierzchołkami powstałych grafów musi znajdować się w przedziale [0-100]. Przerywam działanie.}
  
\end{enumerate}
\end{document}
